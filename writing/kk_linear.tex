\documentclass[11pt,letterpaper]{article}
\usepackage[utf8]{inputenc}

\usepackage{amsmath, amsfonts}
\usepackage{amssymb}
\usepackage{graphicx}
%\usepackage{times}
\usepackage{enumerate}

% Slightly bigger margins than the latex defaults
\usepackage{geometry}
\geometry{verbose,tmargin=3cm,bmargin=3cm,lmargin=2.5cm,rmargin=2.5cm}

%\usepackage[cache=false]{minted}
%\usepackage{minted}

% enable coloring
\usepackage[usenames,dvipsnames]{xcolor}
\newcommand{\highlight}[1]{\textcolor{BrickRed}{#1}}
\newcommand{\ft}[1]{\mathcal{F}[{#1}]}

\usepackage{hyperref}
\hypersetup{
	colorlinks=true,	
	linkcolor=gray, % color of internal links
	urlcolor=blue % color of external links
}

% Prevent overflowing lines due to urls and other hard-to-break entities.
\sloppy

% spacing to slightly comfortable
\usepackage{setspace}
\setstretch{1.3}

%\usepackage[hidealllines=true,backgroundcolor=gray!10]{mdframed}

%\renewcommand{\familydefault}{\sfdefault}

% no page numbers and indents
\setlength{\parindent}{0pt} 
%\pagestyle{empty}


\newcommand{\Gp}{G^{\prime}(\omega)}
\newcommand{\Gpp}{G^{\prime\prime}(\omega)}
\newcommand{\Tp}{G^{\prime}}
\newcommand{\Tpp}{G^{\prime\prime}}

\newcommand{\Gst}{G^{*}(\omega)}



\begin{document}

\begin{center}
\Huge{\textbf{Fitting Discrete Relaxation Spectrum using LASSO}}\\
\end{center}

\tableofcontents

\bigskip

\section{Introduction}

A classic problem in rheology is finding the discrete relaxation spectrum from measurements of the complex modulus $\Gst = \Gp + i \Gpp$. We assume that measurements of $\Gp$ and $\Gpp$ are available at a set of frequencies $\omega = \{\omega_1, \cdots, \omega_n\}$. The goal is to
simultaneously fit $\Gp$ and $\Gpp$ to a set of $N$ discrete Maxwell modes $\mathcal{M} = \{g_j, \tau_j\}$ with $j = 1, \cdots, N$.
\begin{align}
\Gp &  \approx P^{\prime}(\omega) = \sum_{j=1}^{N} g_j \dfrac{\omega^2 \tau_j^2}{1 + \omega^2 \tau_j^2} = \sum_{j=1}^{N} g_j k^{\prime}(\omega \tau_j) \notag\\
\Gpp & \approx P^{\prime\prime}(\omega) = \sum_{j=1}^{N} g_i \dfrac{\omega \tau_j}{1 + \omega^2 \tau_j^2} = \sum_{j=1}^{N} g_j k^{\prime \prime}(\omega \tau_j),
\label{eqn:drs}
\end{align}
where $g_j > 0$ and $\tau_j > 0$ are the modulus and timescale characterizing the $j^\text{th}$ relaxation mode, respectively. $k^{\prime}(z) = z^2/(1+z^2)$ and $k^{\prime\prime}(z) = z/(1+z^2)$ are the kernels corresponding to the storage and loss moduli, respectively.

\medskip

$\mathcal{M}$ is called the discrete relaxation spectrum (DRS). There are specialized programs for computing $\mathcal{M}$ from experimental data. 

\section{Problem}

\begin{itemize}
\item Given a set of $n$ data points $\{\omega_i, G^{\prime}(\omega_i), G^{\prime\prime}(\omega_i)\}$ fit a discrete spectrum by minimizing the error using the ``SMEL Test'' outlined in the Method section.
\begin{equation}
\chi^2 = \dfrac{1}{4 n_d} \sum_{i=1}^{n_d}  \left[w_i^{\prime} \left(D_i^{\prime} - P^{\prime}(\omega_i) \right)^2 + w_i^{\prime\prime} \left(D_i^{\prime\prime} - P^{\prime\prime}(\omega_i) \right)^2 \right],
\label{eqn:chi}
\end{equation}
With LASSO this becomes,
\begin{equation}
\chi^2_\text{LASSO}(\{g_j\}) = \chi^2(\{g_j\}) + \alpha \sum_{j=1}^{N} \left|g_j \right|.
\label{eqn:chilasso}
\end{equation}

\item Compare the properties of the inferred spectrum with that obtained from a standard program like pyReSpect
\end{itemize}


\section{Method}


\begin{enumerate}
\item \textbf{Setup Data and Parameters}
\begin{itemize}
\item Collect experimental observations, $\mathcal{D} = \{\omega_i, D_i^{\prime} = \Tp(\omega_i), D_i^{\prime\prime}=\Tpp(\omega_i)\}$. Stack these moduli into a $2n_d \times 1$ column vector $\mathbf{D}$ so that $\mathbf{D}_{i} =   D_i^{\prime}$ and $\mathbf{D}_{n_d + i} =   D_i^{\prime\prime}$;
\item Denote the boundaries of the frequency window $\omega_{\min} = \min\{\omega_i\}$ and $\omega_{\max} = \max\{\omega_i\}$; mark the boundaries of the modes  $\tau_{\min} = 0.1/\omega_{\max}$ and $\tau_{\max} = 10/\omega_{\min}$ by extending the experimental domain by one decade on either side;
\item Set mode density $\rho_N = 10$ modes/decade. Set the number of modes $N = \rho_N \cdot \text{int}(\log_{10} (\tau_{\max}/\tau_{\min}))$;
\item Set the intermediate timescales $\tau_j$ on a logarithmically equispaced grid via,
\begin{equation}
\dfrac{\tau_j}{\tau_{\min}} = \left(\dfrac{\tau_{\max}}{\tau_{\min}} \right)^{\dfrac{j-1}{N-1}},
\end{equation}
Thus, $\tau_1 = \tau_{\min}$ and $\tau_N = \tau_{\max}$.
\end{itemize}
\item \textbf{Setup for LASSO}
\begin{itemize}
\item Furnish two $n_d \times N$ kernel matrices $\mathbf{K}^{\prime}_{i,j} = k^{\prime}(\omega_i \tau_j)$, and $\mathbf{K}^{\prime\prime}_{i,j} = k^{\prime\prime}(\omega_i  \tau_j)$. Stack $\mathbf{K}^{\prime}$ above $\mathbf{K}^{\prime\prime}$ to produce the $2n_d \times N$ feature matrix $\mathbf{K}$, so that $\mathbf{K}_{i,j} = \mathbf{K}^{\prime}_{i,j}$ and $\mathbf{K}_{n_d+i,j} = \mathbf{K}^{\prime \prime}_{i,j}$;
\item Let $\mathbf{g} = [g_1, \cdots, g_N]^T$ be a column vector of coefficients to be determined so that $\mathbf{D} \approx \mathbf{K g}$ (eqn \ref{eqn:drs});
\item Define a $2n_d \times 2n_d$ diagonal matrix of weights $\mathbf{W}_{ii} = 1/\sqrt{|\mathbf{D}_i|}$ for weighted least-squares;
\item Transform the data vector $\mathbf{D}$ and the feature matrix $\mathbf{K}$ using these weights, $\hat{\mathbf{D}} = \mathbf{W D}$ and $\hat{\mathbf{K}} = \mathbf{W K}$. The least-squares objective function (eqn \ref{eqn:chi}) can be succinctly represented as,
\begin{equation}
\chi^2 = \dfrac{1}{4n_d} (\hat{\mathbf{D}} - \hat{\mathbf{K}} \mathbf{g})^T (\hat{\mathbf{D}} - \hat{\mathbf{K}} \mathbf{g}).
\label{eqn:chi2_unreg}
\end{equation}
The standard unregularized normal equations are $\hat{\mathbf{K}}^T \hat{\mathbf{K}} \mathbf{g} = \hat{\mathbf{K}}^T \mathbf{\hat{D}}$;
\item Use the \texttt{scikit-learn} function \texttt{LassoCV} with three-fold cross-validation to determine an optimal value of $\alpha$ in eqn \ref{eqn:chilasso}. Solve and determine the coefficients $\mathbf{g}$;
\item Assess the quality of the fit using the coefficient of determination, or $R^2$, as a proxy for the quality of the fit. Test if $R^2 \ge 0.95$ (or some other reasonable threshold).
\end{itemize}
\end{enumerate}

\section{Results}


\section{Summary}


\section{References}
%
%\begin{enumerate}[(i)]
%\item Lennon et al., \textit{J. Rheol}, \textbf{64}, 551, (2020); \href{https://doi.org/10.1122/1.5132693}{doi}.
%\item Peiponen and Saarinen, \textit{Rep. Prog. Phys.}, \textbf{72}, 056401, (2009); \href{https://doi.org/10.1088/0034-4885/72/5/056401}{doi}.
%\item Martinetti and Ewoldt, \textit{Phys. Fluids}, \textbf{31}, 021213 (2019); \href{https://doi.org/10.1063/1.5085025}{doi}
%\end{enumerate}

\end{document}
